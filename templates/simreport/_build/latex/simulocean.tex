% Generated by Sphinx.
\def\sphinxdocclass{report}
\documentclass[letterpaper,10pt,openany, oneside]{sphinxmanual}
\usepackage[utf8]{inputenc}
\DeclareUnicodeCharacter{00A0}{\nobreakspace}
\usepackage[T1]{fontenc}
\usepackage[english]{babel}
\usepackage{times}
\usepackage[Sonny]{fncychap}
\usepackage{longtable}
\usepackage{sphinx}
\usepackage{multirow}


\title{Automatic Teakwood Report}
\date{27. May 2013}
\release{}
\author{user name}
\newcommand{\sphinxlogo}{\includegraphics{chart.png}\par}
\renewcommand{\releasename}{Release}
\makeindex

\makeatletter
\def\PYG@reset{\let\PYG@it=\relax \let\PYG@bf=\relax%
    \let\PYG@ul=\relax \let\PYG@tc=\relax%
    \let\PYG@bc=\relax \let\PYG@ff=\relax}
\def\PYG@tok#1{\csname PYG@tok@#1\endcsname}
\def\PYG@toks#1+{\ifx\relax#1\empty\else%
    \PYG@tok{#1}\expandafter\PYG@toks\fi}
\def\PYG@do#1{\PYG@bc{\PYG@tc{\PYG@ul{%
    \PYG@it{\PYG@bf{\PYG@ff{#1}}}}}}}
\def\PYG#1#2{\PYG@reset\PYG@toks#1+\relax+\PYG@do{#2}}

\expandafter\def\csname PYG@tok@gd\endcsname{\def\PYG@tc##1{\textcolor[rgb]{0.63,0.00,0.00}{##1}}}
\expandafter\def\csname PYG@tok@gu\endcsname{\let\PYG@bf=\textbf\def\PYG@tc##1{\textcolor[rgb]{0.50,0.00,0.50}{##1}}}
\expandafter\def\csname PYG@tok@gt\endcsname{\def\PYG@tc##1{\textcolor[rgb]{0.00,0.25,0.82}{##1}}}
\expandafter\def\csname PYG@tok@gs\endcsname{\let\PYG@bf=\textbf}
\expandafter\def\csname PYG@tok@gr\endcsname{\def\PYG@tc##1{\textcolor[rgb]{1.00,0.00,0.00}{##1}}}
\expandafter\def\csname PYG@tok@cm\endcsname{\let\PYG@it=\textit\def\PYG@tc##1{\textcolor[rgb]{0.25,0.50,0.56}{##1}}}
\expandafter\def\csname PYG@tok@vg\endcsname{\def\PYG@tc##1{\textcolor[rgb]{0.73,0.38,0.84}{##1}}}
\expandafter\def\csname PYG@tok@m\endcsname{\def\PYG@tc##1{\textcolor[rgb]{0.13,0.50,0.31}{##1}}}
\expandafter\def\csname PYG@tok@mh\endcsname{\def\PYG@tc##1{\textcolor[rgb]{0.13,0.50,0.31}{##1}}}
\expandafter\def\csname PYG@tok@cs\endcsname{\def\PYG@tc##1{\textcolor[rgb]{0.25,0.50,0.56}{##1}}\def\PYG@bc##1{\setlength{\fboxsep}{0pt}\colorbox[rgb]{1.00,0.94,0.94}{\strut ##1}}}
\expandafter\def\csname PYG@tok@ge\endcsname{\let\PYG@it=\textit}
\expandafter\def\csname PYG@tok@vc\endcsname{\def\PYG@tc##1{\textcolor[rgb]{0.73,0.38,0.84}{##1}}}
\expandafter\def\csname PYG@tok@il\endcsname{\def\PYG@tc##1{\textcolor[rgb]{0.13,0.50,0.31}{##1}}}
\expandafter\def\csname PYG@tok@go\endcsname{\def\PYG@tc##1{\textcolor[rgb]{0.19,0.19,0.19}{##1}}}
\expandafter\def\csname PYG@tok@cp\endcsname{\def\PYG@tc##1{\textcolor[rgb]{0.00,0.44,0.13}{##1}}}
\expandafter\def\csname PYG@tok@gi\endcsname{\def\PYG@tc##1{\textcolor[rgb]{0.00,0.63,0.00}{##1}}}
\expandafter\def\csname PYG@tok@gh\endcsname{\let\PYG@bf=\textbf\def\PYG@tc##1{\textcolor[rgb]{0.00,0.00,0.50}{##1}}}
\expandafter\def\csname PYG@tok@ni\endcsname{\let\PYG@bf=\textbf\def\PYG@tc##1{\textcolor[rgb]{0.84,0.33,0.22}{##1}}}
\expandafter\def\csname PYG@tok@nl\endcsname{\let\PYG@bf=\textbf\def\PYG@tc##1{\textcolor[rgb]{0.00,0.13,0.44}{##1}}}
\expandafter\def\csname PYG@tok@nn\endcsname{\let\PYG@bf=\textbf\def\PYG@tc##1{\textcolor[rgb]{0.05,0.52,0.71}{##1}}}
\expandafter\def\csname PYG@tok@no\endcsname{\def\PYG@tc##1{\textcolor[rgb]{0.38,0.68,0.84}{##1}}}
\expandafter\def\csname PYG@tok@na\endcsname{\def\PYG@tc##1{\textcolor[rgb]{0.25,0.44,0.63}{##1}}}
\expandafter\def\csname PYG@tok@nb\endcsname{\def\PYG@tc##1{\textcolor[rgb]{0.00,0.44,0.13}{##1}}}
\expandafter\def\csname PYG@tok@nc\endcsname{\let\PYG@bf=\textbf\def\PYG@tc##1{\textcolor[rgb]{0.05,0.52,0.71}{##1}}}
\expandafter\def\csname PYG@tok@nd\endcsname{\let\PYG@bf=\textbf\def\PYG@tc##1{\textcolor[rgb]{0.33,0.33,0.33}{##1}}}
\expandafter\def\csname PYG@tok@ne\endcsname{\def\PYG@tc##1{\textcolor[rgb]{0.00,0.44,0.13}{##1}}}
\expandafter\def\csname PYG@tok@nf\endcsname{\def\PYG@tc##1{\textcolor[rgb]{0.02,0.16,0.49}{##1}}}
\expandafter\def\csname PYG@tok@si\endcsname{\let\PYG@it=\textit\def\PYG@tc##1{\textcolor[rgb]{0.44,0.63,0.82}{##1}}}
\expandafter\def\csname PYG@tok@s2\endcsname{\def\PYG@tc##1{\textcolor[rgb]{0.25,0.44,0.63}{##1}}}
\expandafter\def\csname PYG@tok@vi\endcsname{\def\PYG@tc##1{\textcolor[rgb]{0.73,0.38,0.84}{##1}}}
\expandafter\def\csname PYG@tok@nt\endcsname{\let\PYG@bf=\textbf\def\PYG@tc##1{\textcolor[rgb]{0.02,0.16,0.45}{##1}}}
\expandafter\def\csname PYG@tok@nv\endcsname{\def\PYG@tc##1{\textcolor[rgb]{0.73,0.38,0.84}{##1}}}
\expandafter\def\csname PYG@tok@s1\endcsname{\def\PYG@tc##1{\textcolor[rgb]{0.25,0.44,0.63}{##1}}}
\expandafter\def\csname PYG@tok@gp\endcsname{\let\PYG@bf=\textbf\def\PYG@tc##1{\textcolor[rgb]{0.78,0.36,0.04}{##1}}}
\expandafter\def\csname PYG@tok@sh\endcsname{\def\PYG@tc##1{\textcolor[rgb]{0.25,0.44,0.63}{##1}}}
\expandafter\def\csname PYG@tok@ow\endcsname{\let\PYG@bf=\textbf\def\PYG@tc##1{\textcolor[rgb]{0.00,0.44,0.13}{##1}}}
\expandafter\def\csname PYG@tok@sx\endcsname{\def\PYG@tc##1{\textcolor[rgb]{0.78,0.36,0.04}{##1}}}
\expandafter\def\csname PYG@tok@bp\endcsname{\def\PYG@tc##1{\textcolor[rgb]{0.00,0.44,0.13}{##1}}}
\expandafter\def\csname PYG@tok@c1\endcsname{\let\PYG@it=\textit\def\PYG@tc##1{\textcolor[rgb]{0.25,0.50,0.56}{##1}}}
\expandafter\def\csname PYG@tok@kc\endcsname{\let\PYG@bf=\textbf\def\PYG@tc##1{\textcolor[rgb]{0.00,0.44,0.13}{##1}}}
\expandafter\def\csname PYG@tok@c\endcsname{\let\PYG@it=\textit\def\PYG@tc##1{\textcolor[rgb]{0.25,0.50,0.56}{##1}}}
\expandafter\def\csname PYG@tok@mf\endcsname{\def\PYG@tc##1{\textcolor[rgb]{0.13,0.50,0.31}{##1}}}
\expandafter\def\csname PYG@tok@err\endcsname{\def\PYG@bc##1{\setlength{\fboxsep}{0pt}\fcolorbox[rgb]{1.00,0.00,0.00}{1,1,1}{\strut ##1}}}
\expandafter\def\csname PYG@tok@kd\endcsname{\let\PYG@bf=\textbf\def\PYG@tc##1{\textcolor[rgb]{0.00,0.44,0.13}{##1}}}
\expandafter\def\csname PYG@tok@ss\endcsname{\def\PYG@tc##1{\textcolor[rgb]{0.32,0.47,0.09}{##1}}}
\expandafter\def\csname PYG@tok@sr\endcsname{\def\PYG@tc##1{\textcolor[rgb]{0.14,0.33,0.53}{##1}}}
\expandafter\def\csname PYG@tok@mo\endcsname{\def\PYG@tc##1{\textcolor[rgb]{0.13,0.50,0.31}{##1}}}
\expandafter\def\csname PYG@tok@mi\endcsname{\def\PYG@tc##1{\textcolor[rgb]{0.13,0.50,0.31}{##1}}}
\expandafter\def\csname PYG@tok@kn\endcsname{\let\PYG@bf=\textbf\def\PYG@tc##1{\textcolor[rgb]{0.00,0.44,0.13}{##1}}}
\expandafter\def\csname PYG@tok@o\endcsname{\def\PYG@tc##1{\textcolor[rgb]{0.40,0.40,0.40}{##1}}}
\expandafter\def\csname PYG@tok@kr\endcsname{\let\PYG@bf=\textbf\def\PYG@tc##1{\textcolor[rgb]{0.00,0.44,0.13}{##1}}}
\expandafter\def\csname PYG@tok@s\endcsname{\def\PYG@tc##1{\textcolor[rgb]{0.25,0.44,0.63}{##1}}}
\expandafter\def\csname PYG@tok@kp\endcsname{\def\PYG@tc##1{\textcolor[rgb]{0.00,0.44,0.13}{##1}}}
\expandafter\def\csname PYG@tok@w\endcsname{\def\PYG@tc##1{\textcolor[rgb]{0.73,0.73,0.73}{##1}}}
\expandafter\def\csname PYG@tok@kt\endcsname{\def\PYG@tc##1{\textcolor[rgb]{0.56,0.13,0.00}{##1}}}
\expandafter\def\csname PYG@tok@sc\endcsname{\def\PYG@tc##1{\textcolor[rgb]{0.25,0.44,0.63}{##1}}}
\expandafter\def\csname PYG@tok@sb\endcsname{\def\PYG@tc##1{\textcolor[rgb]{0.25,0.44,0.63}{##1}}}
\expandafter\def\csname PYG@tok@k\endcsname{\let\PYG@bf=\textbf\def\PYG@tc##1{\textcolor[rgb]{0.00,0.44,0.13}{##1}}}
\expandafter\def\csname PYG@tok@se\endcsname{\let\PYG@bf=\textbf\def\PYG@tc##1{\textcolor[rgb]{0.25,0.44,0.63}{##1}}}
\expandafter\def\csname PYG@tok@sd\endcsname{\let\PYG@it=\textit\def\PYG@tc##1{\textcolor[rgb]{0.25,0.44,0.63}{##1}}}

\def\PYGZbs{\char`\\}
\def\PYGZus{\char`\_}
\def\PYGZob{\char`\{}
\def\PYGZcb{\char`\}}
\def\PYGZca{\char`\^}
\def\PYGZam{\char`\&}
\def\PYGZlt{\char`\<}
\def\PYGZgt{\char`\>}
\def\PYGZsh{\char`\#}
\def\PYGZpc{\char`\%}
\def\PYGZdl{\char`\$}
\def\PYGZti{\char`\~}
% for compatibility with earlier versions
\def\PYGZat{@}
\def\PYGZlb{[}
\def\PYGZrb{]}
\makeatother

\begin{document}

\maketitle
\tableofcontents
\phantomsection\label{index::doc}
\includegraphics{Teakwood_logo.jpg}




\chapter{Introduction}
\label{introduction:introduction}\label{introduction:automatic-teakwood-report}\label{introduction::doc}
Sample Description here.


\chapter{Background Description}
\label{background:background-description}\label{background::doc}
Under what condition to make such a simulation.
fetch formatted description model.


\chapter{Model}
\label{model:model}\label{model::doc}
Fetch model description from models.

Delft3D, developed by Deltares (formerly Delft Hydraulics), is a flexible integrated modelling suite, which simulates two-dimensional (in either the horizontal or a vertical plane) and three-dimensional flow, sediment transport and morphology, waves, water quality and ecology and is capable of handling the interactions between these processes. After Delft3D-FLOW was open-sourced in 2011, more and more researchers started using Delft3D.


\chapter{Job}
\label{job:job}\label{job::doc}
Fetch job input data from models and render it here.


\chapter{Output}
\label{output:output}\label{output::doc}
date, time : 2013-05-24, 11:34:31
SUMMARY FOR PARTITION : 1
\textbf{* WARNING Thin dam ( 50, 141) lies on an inactive point
0 errors and 1 warnings
returning to main program from domain new02b
------------------------------------------------------------------------------
SUMMARY FOR PARTITION : 2
0 errors and 0 warnings
returning to main program from domain new02b
------------------------------------------------------------------------------
SUMMARY FOR PARTITION : 3
*} WARNING Dry point ( 7, 113) lies on an inactive point
{\color{red}\bfseries{}**}* WARNING Station lies outside the computational domain
0 errors and 2 warnings
returning to main program from domain new02b
---------------------------------------------------------------------

This shows you the available output data

FINISHED Delft3D-FLOW runid : new02b date, time : 2013-05-24, 11:34:31 SUMMARY FOR PARTITION : 1 \textbf{* WARNING Thin dam ( 50, 141) lies on an inactive point 0 errors and 1 warnings returning to main program from domain new02b ------------------------------------------------------------------------------ SUMMARY FOR PARTITION : 2 0 errors and 0 warnings returning to main program from domain new02b ------------------------------------------------------------------------------ SUMMARY FOR PARTITION : 3 *} WARNING Dry point ( 7, 113) lies on an inactive point \textbf{* WARNING Station lies outside the computational domain 0 errors and 2 warnings returning to main program from domain new02b ------------------------------------------------------------------------------ SUMMARY FOR PARTITION : 4 *} WARNING Station lies outside the computational domain 0 errors and 1 warnings returning to main program from domain new02b ------------------------------------------------------------------------------ D\_Hydro {[}1369413271.495435{]} \textgreater{}\textgreater{} d\_hydro shutting down normally D\_Hydro {[}1369413271.495435{]} \textgreater{}\textgreater{} d\_hydro shutting down normally D\_Hydro {[}1369413271.495476{]} \textgreater{}\textgreater{} d\_hydro shutting down normally D\_Hydro {[}1369413271.500990{]} \textgreater{}\textgreater{} d\_hydro shutting down normally
==================   ============
Attribute            Numerical
==================   ============
Tide                 40
Wind                 41
Humidity             42
Tide                 40
Wind                 41
Humidity             42
Tide                 40
Wind                 41
Humidity             42
Tide                 40
Wind                 41
Humidity             42
Tide                 40
==================   ============


\chapter{Data}
\label{data:data}\label{data::doc}
fetch output data and displays it here.


\chapter{Visualization}
\label{visualization:visualization}\label{visualization::doc}
Visualize output data, or observal data to make a comparison.
sample image:

\includegraphics{chart.jpeg}


\chapter{Discussion}
\label{discussion:discussion}\label{discussion::doc}
conduct a simple discussion basing on visualization compare.


\chapter{Sample}
\label{sample:sample}\label{sample::doc}

\section{Sphinx Cheat Sheet}
\label{sample:sphinx-cheat-sheet}\label{sample:sphinx-helpers}
Wherein I show by example how to do some things in Sphinx (you can see
a literal version of this file below in {\hyperref[sample:sphinx-literal]{\emph{This file}}})


\subsection{Making a list}
\label{sample:id1}\label{sample:making-a-list}
It is easy to make lists in rest


\subsubsection{Bullet points}
\label{sample:bullet-points}
This is a subsection making bullet points
\begin{itemize}
\item {} 
point A

\item {} 
point B

\item {} 
point C

\end{itemize}


\subsubsection{Enumerated points}
\label{sample:enumerated-points}
This is a subsection making numbered points
\begin{enumerate}
\item {} 
point A

\item {} 
point B

\item {} 
point C

\end{enumerate}


\subsection{Making a table}
\label{sample:id2}\label{sample:making-a-table}
This shows you how to make a table -- if you only want to make a list
see {\hyperref[sample:making-a-list]{\emph{Making a list}}}.

\begin{tabulary}{\linewidth}{|L|L|}
\hline
\textbf{
Name
} & \textbf{
Age
}\\\hline

John D Hunter
 & 
40
\\\hline

Cast of Thousands
 & 
41
\\\hline

And Still More
 & 
42
\\\hline
\end{tabulary}



\subsection{Making links}
\label{sample:id3}\label{sample:making-links}

\subsubsection{Cross-references sections and documents}
\label{sample:cross-references-sections-and-documents}
Use reST labels to cross-reference sections and other documents. The
mechanism for referencing another reST document or a subsection in any
document, including within a document are identical. Place a
\emph{reference label} above the section heading, like this:

\begin{Verbatim}[commandchars=\\\{\}]
.. \_sphinx\_helpers:

====================
 Sphinx Cheat Sheet
====================
\end{Verbatim}

Note the blank line between the \emph{reference label} and the section
heading is important!

Then refer to the \emph{reference label} in another
document like this:

\begin{Verbatim}[commandchars=\\\{\}]
:ref:{}`sphinx\_helpers{}`
\end{Verbatim}

The reference is replaced with the section title when Sphinx builds
the document while maintaining the linking mechanism.  For example,
the above reference will appear as {\hyperref[sample:sphinx-helpers]{\emph{Sphinx Cheat Sheet}}}.  As the
documentation grows there are many references to keep track of.

For documents, please use a \emph{reference label} that matches the file
name.  For sections, please try and make the \emph{refence label} something
meaningful and try to keep abbreviations limited.  Along these lines,
we are using \emph{underscores} for multiple-word \emph{reference labels}
instead of hyphens.

Sphinx documentation on \href{http://sphinx.pocoo.org/markup/inline.html\#cross-referencing-arbitrary-locations}{Cross-referencing arbitrary locations}
has more details.


\subsubsection{External links}
\label{sample:external-links}
For external links you are likely to use only once, simple include the
like in the text.  This link to \href{http://www.google.com}{google} was
made like this:

\begin{Verbatim}[commandchars=\\\{\}]
{}`google \textless{}http://www.google.com\textgreater{}{}`\_
\end{Verbatim}

For external links you will reference frequently, we have created a
\code{links\_names.txt} file.  These links can then be used throughout the
documentation.  Links in the \code{links\_names.txt} file are created
using the \href{http://docutils.sourceforge.net/docs/user/rst/quickref.html\#hyperlink-targets}{reST reference}
syntax:

\begin{Verbatim}[commandchars=\\\{\}]
.. \_targetname: http://www.external\_website.org
\end{Verbatim}

To refer to the reference in a separate reST file, include the
\code{links\_names.txt} file and refer to the link through it's target
name.  For example, put this include at the bottom of your reST
document:

\begin{Verbatim}[commandchars=\\\{\}]
.. include:: ../links\_names.txt
\end{Verbatim}

and refer to the hyperlink target:

\begin{Verbatim}[commandchars=\\\{\}]
blah blah blah targetname\_ more blah
\end{Verbatim}


\subsubsection{Links to classes, modules and functions}
\label{sample:links-to-classes-modules-and-functions}
You can also reference classes, modules, functions, etc that are
documented using the sphinx \href{http://sphinx.pocoo.org/ext/autodoc.html}{autodoc} facilites.  For example,
see the module \code{matplotlib.backend\_bases} documentation, or the
class \code{LocationEvent}, or the method
\code{mpl\_connect()}.


\subsection{ipython sessions}
\label{sample:ipython-highlighting}\label{sample:ipython-sessions}
Michael Droettboom contributed a sphinx extension which does pygments
syntax highlighting on ipython sessions

\begin{Verbatim}[commandchars=\\\{\}]
In [69]: lines = plot([1,2,3])

In [70]: setp(lines)
  alpha: float
  animated: [True \textbar{} False]
  antialiased or aa: [True \textbar{} False]
  ...snip
\end{Verbatim}

This support is included in this template, but will also be included
in a future version of Pygments by default.


\subsection{Formatting text}
\label{sample:id4}\label{sample:formatting-text}
You use inline markup to make text \emph{italics}, \textbf{bold}, or \code{monotype}.

You can represent code blocks fairly easily:

\begin{Verbatim}[commandchars=\\\{\}]
\PYG{k+kn}{import} \PYG{n+nn}{numpy} \PYG{k+kn}{as} \PYG{n+nn}{np}
\PYG{n}{x} \PYG{o}{=} \PYG{n}{np}\PYG{o}{.}\PYG{n}{random}\PYG{o}{.}\PYG{n}{rand}\PYG{p}{(}\PYG{l+m+mi}{12}\PYG{p}{)}
\end{Verbatim}

Or literally include code:


\subsection{Using math}
\label{sample:id5}\label{sample:using-math}
In sphinx you can include inline math $x\leftarrow y\ x\forall
y\ x-y$ or display math
\begin{gather}
\begin{split}W^{3\beta}_{\delta_1 \rho_1 \sigma_2} = U^{3\beta}_{\delta_1 \rho_1} + \frac{1}{8 \pi 2} \int^{\alpha_2}_{\alpha_2} d \alpha^\prime_2 \left[\frac{ U^{2\beta}_{\delta_1 \rho_1} - \alpha^\prime_2U^{1\beta}_{\rho_1 \sigma_2} }{U^{0\beta}_{\rho_1 \sigma_2}}\right]\end{split}\notag
\end{gather}
This documentation framework includes a Sphinx extension,
\code{sphinxext/mathmpl.py}, that uses matplotlib to render math
equations when generating HTML, and LaTeX itself when generating a
PDF.  This can be useful on systems that have matplotlib, but not
LaTeX, installed.  To use it, add \code{mathpng} to the list of
extensions in \code{conf.py}.

Current SVN versions of Sphinx now include built-in support for math.
There are two flavors:
\begin{itemize}
\item {} 
pngmath: uses dvipng to render the equation

\item {} 
jsmath: renders the math in the browser using Javascript

\end{itemize}

To use these extensions instead, add \code{sphinx.ext.pngmath} or
\code{sphinx.ext.jsmath} to the list of extensions in \code{conf.py}.

All three of these options for math are designed to behave in the same
way.


\subsection{Inserting matplotlib plots}
\label{sample:inserting-matplotlib-plots}
Inserting automatically-generated plots is easy.  Simply put the script to
generate the plot in any directory you want, and refer to it using the \code{plot}
directive.  All paths are considered relative to the top-level of the
documentation tree.  To include the source code for the plot in the document,
pass the \code{include-source} parameter:

\begin{Verbatim}[commandchars=\\\{\}]
.. plot:: devel/guidelines/elegant.py
   :include-source:
\end{Verbatim}

In the HTML version of the document, the plot includes links to the
original source code, a high-resolution PNG and a PDF.  In the PDF
version of the document, the plot is included as a scalable PDF.


\subsection{Emacs helpers}
\label{sample:emacs-helpers}
See \emph{rst\_emacs}


\subsection{Inheritance diagrams}
\label{sample:inheritance-diagrams}
Inheritance diagrams can be inserted directly into the document by
providing a list of class or module names to the
\code{inheritance-diagram} directive.

For example:

\begin{Verbatim}[commandchars=\\\{\}]
.. inheritance-diagram:: codecs
\end{Verbatim}

produces:


\subsection{This file}
\label{sample:this-file}\label{sample:sphinx-literal}
© Copyright 2014, The Teakwood deployment and visualization framework for coastal modeling is © 2014 the Coastal Hazards Research Collaboratory and distributed under the GNU Lesser General Public License (LGPL) v 3.0. This website makes use of the wonderful icon sets by Studio MX and FAMFAMFAM. The icons used are distributed following the original license respectively. The computational resources are provided by the Coastal Hazards Research Collaboratory, the Center for Computational and Technology, and the Louisiana Optical Network Initiative. Created using Sphinx 1.1.3.



\renewcommand{\indexname}{Index}
\printindex
\end{document}
